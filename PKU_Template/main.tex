\documentclass{beamer}
\usetheme{Warsaw}
\useinnertheme{default}
\useoutertheme{infolines}
\graphicspath{{figures/}}


% ========================
% Packages
% ========================
\usepackage{
	amsmath,        % math environment
	amssymb,		% extended symbols
	enumerate,	    % enumerate environment
	extarrows,      % special arrows
	graphicx,		% include figures
	lastpage,		% lastpage
	multicol,		% multi-column for table
	multirow,		% multi-row for table
	palatino,       % for font
	xeCJK,          % chinese
}


% ========================
% Colors
% ========================
\definecolor{MyOrange}{RGB}{212,69,0}
\definecolor{MyGray}{RGB}{132,131,130}
\definecolor{MyRed}{RGB}{139,0,18}
\definecolor{MyYellow}{RGB}{255,204,0}
\setbeamercolor{title}{fg=MyGray}
\setbeamercolor{structure}{fg=MyRed}
\setbeamercolor{local structure}{fg=black}
\setbeamercolor{section in head/foot}{bg=MyGray,fg=MyRed}
\setbeamercolor{author in head/foot}{bg=MyGray}
%\setbeamercolor{date in head/foot}{fg=MyYellow}


% ========================
% Fonts & Layouts
% ========================
\usefonttheme{serif}
%\usepackage{palatino}
\setbeamerfont{title like}{shape=\scshape}
\setbeamerfont{frametitle}{shape=\scshape}
\setbeamertemplate{itemize items}[circle]
\setbeamertemplate{enumerate items}[default]
\logo{\includegraphics[scale=0.1]{PKU.png}}
\setbeamercovered{transparent}
%\setbeamertemplate{footline}[frame number]
%\expandafter\def\expandafter\insertshorttitle\expandafter{%
%	\insertshorttitle\hfill%
%	\insertframenumber\,/\,\inserttotalframenumber}


% ========================
% Commands
% ========================
\renewcommand{\thefootnote}{}
\renewcommand{\today}{\number\year 年\number\month 月\number\day 日}

% ========================
% Title Page
% ========================
\title{\textcolor{white}{一个简单粗暴的模板}}
\subtitle{\textcolor{white}{一个使用了北大红的简单粗暴的模板}}  
\author{这是我:杨凯欣}
\date{\today} 


% ========================
% Content
% ========================
\begin{document}
	
	
	% ========================
	% Title Page
	% ========================
	\begin{frame}
	\titlepage
\end{frame}


% ========================
% Content
% ========================
\begin{frame}
\tableofcontents
\end{frame}


% ========================
% First Section
% ========================
\section{第一节(揉...揉天应穴?误...)}


% ------------------------
% Section 1.1
% ------------------------
\subsection{这是第一节的第一小节}
\begin{frame}{为了省事就用英文了...}
xxxx reads the original verilog file line by line, and parses each line by some key words. \\
They define new structures called \textcolor{MyRed}{\bf `graph', `vertex' and `edge'}, which are corresponding to different elements in the netlist:
$$\text{netlist}\xLongrightarrow{\text{converted to}}\text{graph}$$ 
$$\text{gates/FFs}\xLongrightarrow{\text{converted to}}\text{vertices}$$ 
$$\text{inputs \& outputs of each vertex}\xLongrightarrow{\text{converted to}}\text{edges}$$ 
\footnote{\tiny Reference:}
\end{frame}


% ------------------------
% Section 1.2
% ------------------------
\subsection{这是第一节的第二小节}
\begin{frame}
\begin{block}{Two steps to identify all state FFs (sequential vertices):}
\begin{enumerate}
\item Identifying all the FFs, ...
\item Searching for cycles in the graph, ... 
\end{enumerate}
\end{block}

\begin{block}{使用北大标识要遵守学校规范哦(其实我也不知道自己用的对不对...)}
Ref: https://vim.pku.edu.cn/xzzq/index.htm
\begin{figure}[H]
\includegraphics[width=0.25\columnwidth]{PKU.png}
\caption{PKU}
\label{PKU}
\end{figure}
\end{block}

\end{frame}


% ========================
% Second Section
% ========================
\section{第二节(挤...挤按睛明穴...)}


% ------------------------
% Section 2.1
% ------------------------
\subsection{这是第二节的第一小节}
\begin{frame}
\begin{enumerate}[Q1:]
\item How many gates inserted? \\
- Decided by users (variable in the configuration file)
\item Where to insert the gates? \\
- The outputs of combinational gates 
\end{enumerate}
{\bf Conclusion: xxxxxxxxxxx}
\end{frame}


% ========================
% Last Page
% ========================
\section{}
\begin{frame}
\begin{center} 
{\bfseries \Large Thank you!} \\
任何问题都可以联系我:) \\
email: kaixinya@usc.edu
\end{center}
\end{frame}


% ========================
% End
% ========================
\end{document}


% ========================================================================
% ========================================================================
% ========================================================================